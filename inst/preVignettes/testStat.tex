\documentclass[12pt]{article}
\parindent 0cm
\usepackage{amssymb,amsmath,graphicx,float}
\usepackage[UKenglish]{isodate}
\usepackage[round]{natbib}
\newcommand{\pt}{\mbox{\scalebox{0.4}{$\bullet$}}}
\newcommand{\Cov}{\mbox{$\textup{\textrm{Cov}}$}}
\newcommand{\Var}{\mbox{$\textup{\textrm{Var}}$}}
\newcommand{\tils}{\mbox{$\tilde{s}$}}
\newcommand{\tilsig}{\mbox{$\tilde{\sigma}$}}
\newcommand{\sco}{\mbox{$\cal N$}}
\allowdisplaybreaks
\begin{document}
\title{Mathematical details of the test statistics used by \texttt{kanova}}
\author{Rolf Turner}
\date{\today}
\maketitle

\section{Introduction}
\label{sec:intro}
This note presents in some detail the formulae for the test
statistics used by the \texttt{kanova()} function from the
\texttt{kanova} package.  These statistics are based on, and
generalise, the ideas discussed in \cite{DiggleEtAl2000} and in
\cite{Hahn2012}.  (See also \cite{DiggleEtAl1991}.)

The statistics consist of sums of integrals (over the argument $r$
of the $K$-function) of the usual sort of analysis of variance
``regression'' sums of squares, down-weighted over $r$ by the
estimated variance of the quantities being squared.  The limits
of integration $r_0$ and $r_1$ \emph{could} be specified in
the software (e.g. in the related \texttt{spatstat} function
\texttt{studpermu.test()} they can be specified in the argument
\texttt{rinterval}).  However there is currently no provision
for this in \texttt{kanova()}, and $r_0$ and $r_1$ are taken
to be the min and max of the $r$ component of the \texttt{"fv"}
object returned by \texttt{Kest()}.   Usually $r_0$ is 0 and $r_1$
is $1/4$ of the length of the shorter side of the bounding box of
the observation window in question.

There are test statistics for:
\begin{itemize}
\item one-way analysis of variance (one grouping factor),
\item main effects in a two-way (two grouping factors) additive model, and
\item a model with interaction versus an additive model in a two-way
context.
\end{itemize}

\section{The data}
In the context of a single classification factor A, with $a$ levels,
the data consist of $K$-functions $K_{ij}(r)$, $i = 1, \ldots,
a$, $k = 1, \ldots, n_i$.  The function $K_{ij}(r)$ is constructed
(estimated) from an observed point pattern $X_{ij}$.

In the context of two classification factors A and B, with
$a$ levels and $b$ levels respectively, the data consist of
$K$-functions $K_{ijk}(r)$, $i = 1, \ldots, a$, $j = 1, \ldots, b$,
$k = 1, \ldots, n_{ij}$.  The function $K_{ijk}(r)$ is constructed
(estimated) from an observed point pattern $X_{ijk}$.

\subsection{Modelling heteroscedasticity}
\label{sec:heteroMod}

In order to model the variances of the $K$-functions that appear
in the current context, we need to make the following simplifying
assumptions about of the nature of these variances:
\begin{enumerate}
\item with each cell of the model there is associated an underlying
variance function denoted by $\sigma_i^2(r)$ ($i = 1, \ldots, a$)
in the case of a one way design, and by $\sigma_{ij}^2(r)$ ($i =
1, \ldots, a$, $j = 1, \ldots, b$) in the case of a two way design

\item with each observed pattern there is associated a (positive)
weight, depending on the number of points in the pattern and denoted
by $w_{ij}$ in the case of a one way design and by $w_{ijk}$ in
the case of a two way design
\end{enumerate}

In terms of these assumptions, we assume that the variances of the
$K$-functions are given by:
\begin{align*}
\Var(K_{ij})(r) &= \sigma_i^2(r)/w_{ij} \mbox{~for a one way design}\\
\Var(K_{ijk})(r) &= \sigma_{ij}^2(r)/w_{ijk} \mbox{~for a two way design}
\end{align*}
This model has no theoretical basis and is justified only
by its simplicity and intuitive appeal.

In the single classification context the weight associated
with $K_{ij}(r)$ is specified to be $w_{ij} = m_{ij}^{\eta}$
where $m_{ij}$ is the number of points in the pattern $X_{ij}$.
In the context of two classification factors, the weight associated
with $K_{ijk}(r)$ is specified to be $w_{ijk} = m_{ijk}^{\eta}$
where $m_{ijk}$ is the number of points in the pattern $X_{ijk}$.
The exponent $\eta$ is taken to be a constant, to be specified
by the user of the \texttt{kanova} package.  In the code $\eta$
is denoted by \texttt{expo}.

\subsection{Homoscedasticity is better}

After the package was completed, simulation experiments revealed
that expressing the variances in terms of the proposed weights
appears to be counter-productive.  The power of the tests in
questions seems to be maximised when all the weights are 1,
i.e. when $\eta$ (\texttt{expo}) is 0.  More importantly, the null
distribution of the $p$-values of the tests appears to deviate from
the theoretical ideal, i.e. the $U[0,1]$ distribution, unless the
value of \texttt{expo} is 0.  Hence the tests appear to be valid
\emph{only} if \texttt{expo = 0}.  (More detail will be available
in \cite{DiggleTurner2025}.)

The code of the package was designed so as to do all of the
relevant calculations in terms of the aforementioned weights.
To keep life simple (and to allow for the remote possibility that
in some circumstances the use of weights might be called for) the
code (and the forthcoming exposition in this vignette) have been
left expressed in terms of weights.  However the default value
of \texttt{expo} has been set equal to 0.  Hence, unless the user
explicitly changes the value of \texttt{expo} from its default, all
of the computations will actually be carried out in an un-weighted
manner.  I.e. the data will be treated as being homoscedastic.

\section{Some details about the weighted means of the observations}
\label{sec:wtdMeans}

The test statistics used are (in general) calculated in terms of
various weighted means of the observed $K$-functions.  Explicitly
we define:
\begin{align*}
\tilde{K}_{i\pt}(r)      &= \frac{1}{w_{i\pt}} \sum_{j=1}^{n_i} w_{ij} K_{ij}(r)\\
\tilde{K}_{\pt\pt}(r)    &= \frac{1}{w_{\pt\pt}} \sum_{i=1}^a
                            \sum_{j=1}^{n_i} w_{ij} K_{ij}(r)\\
                         &= \frac{1}{w_{\pt\pt}} \sum_{i=1}^a w_{i\pt}
                                                \tilde{K}_{i\pt}(r) \\
\tilde{K}_{ij \pt}(r) &= \sum_{k=1}^{n_{ij}}
                            \frac{w_{ijk}}{w_{ij \pt}} K_{ijk}(r) \\
\tilde{K}_{i \pt \pt}(r)
              &= \sum_{j=1}^{b} \frac{w_{ij \pt}}{w_{i \pt \pt}}
                   \tilde{K}_{ij \pt}(r) \\
              &= \frac{1}{w_{i \pt \pt}} \sum_{j=1}^{b}
                          \sum_{k=1}^{n_{ij}} w_{ijk} K_{ijk}(r) \\
\tilde{K}_{\pt j \pt}(r)
              &= \sum_{i=1}^{a} \frac{w_{ij \pt}}{w_{\pt j \pt}}
                   \tilde{K}_{ij \pt}(r) \\
              &= \frac{1}{w_{\pt j \pt}} \sum_{i=1}^{a}
                          \sum_{k=1}^{n_{ij}} w_{ijk} K_{ijk}(r) \mbox{~and} \\
\tilde{K}_{\pt \pt \pt}(r)
              &= \sum_{i=i}^a \frac{w_{i \pt \pt}}{w_{\pt \pt \pt}}
                 \tilde{K}_{i \pt \pt}(r) \\
              &= \sum_{j=1}^b \frac{w_{\pt j \pt}}{w_{\pt \pt \pt}}
                 \tilde{K}_{\pt j \pt}(r) \\
              &= \sum_{i=1}^a \sum_{j=1}^{b} \frac{w_{ij \pt}}{w_{\pt \pt \pt}}
                 \tilde{K}_{ij \pt}(r)\\
              &= \sum_{i=1}^a \sum_{j=1}^{b} \sum_{k=1}^{n_{ij}}
                 \frac{w_{ijk}}{w_{\pt \pt \pt}} K_{ijk}(r)
\end{align*}

\section{Estimating the variance functions}
\label{sec:estvarfns}

The variances of the $K$-functions are assumed to be proportional
to functions which are constant over indices within each cell
of the model.  In the context of a single classification factor,
the variance of $K_{ij}(r)$ is taken to be $\sigma_i^2(r)/w_{ij}$.
Under the null hypothesis of ``no A effect'', it is assumed that
the functions $\sigma_i^2(r)$ are all equal to a single function,
$\tilde{\sigma}^2(r)$.  I.e. they do not vary with $i$.  This function is
estimated by
\[
s^2(r) = \frac{1}{n_{\pt} - a} \sum_{i=1}^a \sum_{j=1}^{n_i}
         w_{ij}(K_{ij}(r) - \tilde{K}_{i\pt}(r))^2 \; .
\]
Under the null hypothesisi, this is an unbiased estimate of $\tilde{\sigma}^2(r)$.

In the context of two classification factors, the variance of
$K_{ijk}(r)$ is taken to be $\sigma_{ij}^2(r)/w_{ijk}$.  If we
are testing for an A effect, allowing for a B effect, it is assumed
that, under the null hypothesis, the functions $\sigma_{ij}^2(r)$
do not vary with $i$, and for each $j$ are all equal to a single
function $\tilde{\sigma}_j^2(r)$ (depending only on the B effect).  These
$\tilde{\sigma}^2_j(r)$) are estimated by
\[
s^2_j(r) = \frac{1}{n_{\pt j}} \sum_{i=1}^a \sum_{k=1}^{n_{ij}}
                               w_{ijk}(K_{ijk}(r) - \tilde{K}_{ij\pt}(r))^2 \; .
\]
Under the null hypothesis these are unbiased estimates of the $\tilde{\sigma}^2_j(r)$.

In the context of two classification factors, where we are testing for
interaction against an additive model (unlikely to arise as these circumstances
may be) we need estimates of $\sigma^2_{ij}(r)$.  These are given by
\[
s^2_{ij}(r) = \frac{1}{n_{ij} - 1} \sum_{k=1}^{n_{ij}} w_{ijk}(K_{ijk}(r)
                                            - \tilde{K}_{ij\pt})^2 \; .
\]
These are unbiased estimates of the $\sigma^2_{ij}(r)$.

\section{The test statistics}
\label{sec:teststats}

The test statistics are (numerical) integrals of certain sums of
squares, possibly divided by ``normalisation" or ``homogenisation''
coefficients.  The ``normalisation" is analogous to the
studentisation procedure used by \cite{Hahn2012}.

\subsection{Single classification factor}
\label{sec:teststats.scf}

In the setting of a single classification factor A, the statistic
for testing for an A effect is
\[
T = \sum_{i=1}^a n_i \int_{r_0}^{r_1} (\tilde{K}_i(r) - 
                 \tilde{K}(r))^2/\sco_i(r) \; dr
\]
where $\sco_i(r)$ is the normalisation coefficient.  If the
\texttt{divByVar} argument of \texttt{kanova()} is \texttt{TRUE},
then $\sco_i(r)$ is equal to the estimated variance of $\tilde{K}_i(r)
- \tilde{K}(r)$ which is given by
\[
s^2(r) \left ( \frac{1}{w_i \pt} - \frac{1}{w_{\pt \pt}} \right ) \; .
\]
If \texttt{divByVar} is \texttt{FALSE} then $\sco_i(r)$ is taken to
be identically equal to 1.

\subsection{Two classification factors, testing for A ``allowing for B''}
\label{sec:teststats.testForA}
In the setting of two classification factors A and B, the statistic
for testing for an A effect allowing for a B effect is
\[
T_A = \sum_{i=1}^a n_{i \pt} \int_{r_0}^{r_1} (\tilde{K}_{i \pt \pt}(r) -
                 \tilde{K}_{\pt \pt \pt}(r))^2/\sco_i(r) \; dr
\]
where $\sco_i(r)$ is the normalisation coefficient.  If
\texttt{divByVar} is \texttt{TRUE}, then $\sco_i(r)$ is equal
to the estimated variance of $\tilde{K}_{i \pt \pt}(r) - \tilde{K_{\pt \pt \pt}}(r)$
which is given by
\[
\tils_i^2(r) \left ( \frac{1}{w_{i \pt \pt}} - \frac{2}{w_{\pt \pt \pt}}
                      \right ) + \frac{1}{w_{\pt \pt \pt}} \sum_{\ell=1}^a
                      \frac{w_{i \pt \pt}}{w_{\pt \pt \pt}} \tils_{\ell}^2(r) \; .
\]
The foregoing expression may be re-written, more compactly, and in a form which
makes it more obvious that the quantity is positive, as:
\[
\sco_i(r) = \frac{1}{w_{\pt \pt \pt}} \left [
            \sum_{\ell=1}^a \zeta_{i \ell} \times \tils^2_{\ell}(r) \right ]
\]
where
\begin{align*}
\tils^2_{\ell}(r) &= \sum_{j=1}^b \frac{w_{\ell j\pt}}{w_{\ell \pt \pt}} s_j^2(r),
\; \ell = 1, \ldots, a, \\
\zeta_{i \ell} &= \left \{ \begin{array}{cl}
               \nu_{\ell} & \ell \neq i \\
               \frac{(\nu_i - 1)^2}{\nu_i} & \ell = i 
               \end{array} \right . \\
\nu_{\ell} &= \frac{w_{\ell \pt \pt}}{w_{\pt \pt \pt}}, \;
\ell = 1, \ldots, a.
\end{align*}
If \texttt{divByVar} is \texttt{FALSE} then $\sco_i(r)$ is taken to
be identically equal to 1.

\subsection{Two classification factors, testing for interaction}
\label{sec:teststats.testForInterac}

In the setting in which there are two classification factors and
we are testing for interaction, against an additive models, the
test statistic is
\[
T_{AB} = \sum_{i=1}^a \sum_{j=1}^b n_{ij} \int_{r_0}^{r_1} (\tilde{K}_{ij \pt}(r) -
   \tilde{K}_{i \pt \pt}(r) - \tilde{K}_{\pt j \pt}(r) +
   \tilde{K}_{\pt \pt \pt}(r))^2/\sco_{ij}(r) \; dr
\]
where $\sco_{ij}(r)$ is the normalisation coefficient.  If
\texttt{divByVar} is \texttt{TRUE}, then $\sco_{ij}(r)$ is equal to
the (sample) variance of $\tilde{K}_{ij\pt}(r) - \tilde{K}_{i\pt \pt}(r)
- \tilde{K}_{\pt j \pt}(r) + \tilde{K}_{\pt \pt \pt}(r)$.  The expression for
$\sco_{ij}(r)$, when \texttt{divByVar} is \texttt{TRUE}, is
even messier than the corresponding expression for $\sco_{i}(r)$.
It is given by
\begin{equation}
\label{eq:sampvarSSR}
\begin{split}
s_{ij}^2(r) &\left ( \frac{1}{w_{i j \pt}} -
                                      \frac{2}{w_{i \pt \pt}} -
                                      \frac{2}{w_{\pt j \pt}} +
                                      \frac{2w_{ij \pt}}{w_{i \pt \pt} w_{\pt j \pt}} +
                                      \frac{2}{w_{\pt \pt \pt}} \right )+ \\
               \tils^2_{i \pt}(r) &\left ( \frac{1}{w_{i \pt \pt}} -
                                   \frac{2}{w_{\pt \pt \pt}} \right )+
               \tils^2_{\pt j}(r) \left (\frac{1}{w_{\pt j \pt}} -
                                   \frac{2}{w_{\pt \pt \pt}} \right )+
                            \frac{\tilde{s}^2(r)}{w_{\pt \pt \pt}}
\end{split}
\end{equation}
where
\begin{equation}
\label{eq:sampvars}
\begin{split}
\tils_{i \pt}^2(r) &= \sum_{j=1}^b \frac{w_{ij \pt}}{w_{i \pt \pt}} s^2_{ij}(r) \\
\tils_{\pt j}^2(r) &= \sum_{i=1}^a \frac{w_{ij \pt}}{w_{\pt j \pt}} s^2_{ij}(r)
                        \mbox{~and}\\
\tils^2(r) &= \sum_{i=1}^a \sum_{j=1}^b \frac{w_{i j \pt}}{w_{\pt \pt \pt}}
                                          s^2_{ij}(r) \; .
\end{split}
\end{equation}
Note that \eqref{eq:sampvarSSR} is just \eqref{eq:popvarSSR},
and \eqref{eq:sampvars} is just \eqref{eq:popvars} (see below)
with population quantities replaced by sample (estimated) quantities.

Here are some (terse) details about the variance of
\mbox{$\tilde{K}_{ij \pt}(r) - \tilde{K}_{i \pt \pt}(r)
- \tilde{K}_{\pt j \pt}(r) + \tilde{K}(r)$} as given by
\eqref{eq:popvarSSR}.
\begin{align*}
\Var(\tilde{K}_{ij \pt}(r))      &= \frac{\sigma_{ij}^2(r)}{w_{i j \pt}} \\
\Var(\tilde{K}_{i \pt \pt}(r))   &= \frac{\tilsig_{i\pt}^2(r)}{w_{i \pt \pt}} \\
\Var(\tilde{K}_{\pt j \pt}(r)    &= \frac{\tilsig_{\pt j}^2(r)}{w_{\pt j \pt}} \\
\Var(\tilde{K}_{\pt \pt \pt}(r)) &= \frac{\tilsig^2(r)}{w_{\pt \pt \pt}} \\
\Cov(\tilde{K}_{ij \pt}(r),\tilde{K}_{i \pt \pt})
                                 &= \frac{\sigma_{ij}^2(r)}{w_{i \pt \pt}} \\
\Cov(\tilde{K}_{ij \pt}(r),\tilde{K}_{\pt j \pt})
                                 &= \frac{\sigma_{ij}^2(r)}{w_{\pt j \pt}} \\
\Cov(\tilde{K}_{ij \pt}(r),\tilde{K}_{\pt \pt \pt})
                                 &= \frac{\sigma_{ij}^2(r)}{w_{\pt \pt \pt}} \\
\Cov(\tilde{K}_{i \pt \pt}(r),\tilde{K}_{\pt j \pt})
                                 &= \frac{w_{i j \pt} \sigma_{ij}^2(r)}{w_{i \pt \pt}
                                                             w_{\pt j \pt}} \\
\Cov(\tilde{K}_{i \pt \pt}(r),\tilde{K}_{\pt \pt \pt})
                                 &= \frac{\tilsig_{i \pt}^2(r)}{w_{\pt \pt \pt}} \\
\Cov(\tilde{K}_{\pt j \pt}(r),\tilde{K}_{\pt \pt \pt}(r)
                                 &= \frac{\tilsig_{\pt j}^2(r)}{w_{\pt \pt \pt}}
\end{align*}
where
\begin{equation}
\label{eq:popvars}
\begin{split}
\tilsig_{i \pt}^2(r) &= \sum_{j=1}^b \frac{w_{ij \pt}}{w_{i \pt \pt}} \sigma^2_{ij}(r) \\
\tilsig_{\pt j}^2(r) &= \sum_{i=1}^a \frac{w_{ij \pt}}{w_{\pt j \pt}} \sigma^2_{ij}(r)
                        \mbox{~and}\\
\tilsig^2(r) &= \sum_{i=1}^a \sum_{j=1}^b \frac{w_{i j \pt}}{w_{\pt \pt \pt}}
                                          \sigma^2_{ij}(r) \; .
\end{split}
\end{equation}

Sample calculation:  to see that $\Cov(\tilde{K}_{ij \pt}(r),
\tilde{K}_{i \pt \pt}) = \sigma_{ij}^2/w_{i \pt \pt}$, note that
$\tilde{K}_{i \pt \pt}(r)$ is a weighted sum over $\ell$,
of terms $\tilde{K}_{i \ell \pt}(r)$.% (see page \pageref{pg:kays}).
The $K$-functions involved correspond to
independent patterns, and so are likewise independent.  Consequently
$\tilde{K}_{ij \pt}(r)$ is independent of
$\tilde{K}_{i \ell \pt}(r)$, and the corresponding covariances are 0,
except when $\ell = j$.  We thus get only a single non-zero term from
the sum of the covariances, explicitly
\[
\Cov(\tilde{K}_{ij \pt}(r),\frac{w_{ij \pt}}{w_{i \pt \pt}} \tilde{K}_{ij \pt})
= \frac{w_{ij \pt}}{w_{i \pt \pt}} \Var(\tilde{K}_{ij \pt})
= \frac{w_{ij \pt}}{w_{i \pt \pt}} \frac{\sigma_{ij}^2}{w_{ij \pt}}
= \frac{\sigma_{ij}^2}{w_{i \pt \pt}} \; .
\]

Finally we can obtain the variance term of interest, which is
$\Var(\tilde{K}_{ij \pt}(r) - \tilde{K}_{i \pt \pt}(r) - \tilde{K}_{j
\pt \pt}(r) + \tilde{K}_{\pt \pt \pt}(r))$.  This expression is equal to
\begin{align*}
& \Var(\tilde{K}_{ij \pt}(r)) +
  \Var(\tilde{K}_{i \pt \pt}(r)) +
  \Var(\tilde{K}_{\pt j \pt}(r)) +
  \Var(\tilde{K}_{\pt \pt \pt}(r)) \\
& -2 \Cov(\tilde{K}_{ij \pt}(r),\tilde{K}_{i \pt \pt}(r))
  -2 \Cov(\tilde{K}_{ij \pt}(r),\tilde{K}_{\pt j \pt(r)})
  +2 \Cov(\tilde{K}_{ij \pt}(r),\tilde{K}_{\pt \pt \pt}(r)) \\ 
& +2 \Cov(\tilde{K}_{i \pt \pt}(r),\tilde{K}_{\pt j \pt})
  -2 \Cov(\tilde{K}_{i \pt \pt}(r),\tilde{K}_{\pt \pt \pt}(r))\\
& -2 \Cov(\tilde{K}_{\pt j \pt}(r),\tilde{K}_{\pt \pt \pt}(r)) \; .
\end{align*}
Collecting terms in the foregoing expression, and using the
previously stated symbolic representations of these terms, we obtain
\begin{equation}
\label{eq:popvarSSR}
\begin{split}
&\sigma_{ij}^2(r) \left (
\frac{1}{w_{i j \pt}} -
\frac{2}{w_{i \pt \pt}} -
\frac{2}{w_{\pt j \pt}} +
\frac{2 w_{i j \pt}}{w_{i \pt \pt} w_{\pt j \pt}} +
\frac{2}{w_{\pt \pt \pt}} \right ) + \\
&\tilsig_{i \pt}(r) \left(
\frac{1}{w_{i \pt \pt}} -
\frac{2}{w_{\pt \pt \pt}} \right ) +
\tilsig_{\pt j}(r) \left(
\frac{1}{w_{\pt j \pt}} -
\frac{2}{w_{\pt \pt \pt}} \right ) +
\frac{\tilsig(r)}{w_{\pt \pt \pt}} \; .
\end{split}
\end{equation}

Replacing the population variances by their corresponding estimates
(sample quantities) we obtain \eqref{eq:sampvarSSR}.
\bibliographystyle{plainnat}
\bibliography{kanova}
\end{document}
